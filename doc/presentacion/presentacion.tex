\documentclass{beamer}

\usepackage[utf8]{inputenc}
\usepackage[spanish,activeacute]{babel}
\usepackage{listings} % Trozos de código
\usepackage{graphicx} % Imágenes
\usepackage{fancyhdr} % Cabeceras
\usepackage{listings}

\usetheme{Darmstadt}
\usefonttheme{structuresmallcapsserif} 


\title{Desarrollo de un Pong en C con SDL}
\author{David Saltares Márquez \and Alberto Cejas Sánchez}


\AtBeginSection[]
{
  \begin{frame}<beamer>
    \frametitle{Índice}
    \tableofcontents[currentsection,currentsubsection]
  \end{frame}

}

\begin{document}

\begin{frame}
	\titlepage
	
    \begin{picture}(0,0)
        \put(10,-5){\includegraphics[scale=0.1]{img/advuca.png}}
    \end{picture}
    
    \begin{center}
        \includegraphics[scale=1]{img/cc.png}
    \end{center}    
\end{frame}

\begin{frame}
	\frametitle{Índice de contenidos}
	\tableofcontents
\end{frame}

\section{ADVUCA}

\begin{frame}
    \frametitle{¿Pero qué es la ADVUCA?}
    
    \begin{center}
        \begin{alertblock}{¡Es un acrónimo!}
            \emph{ADVUCA} $\rightarrow$ Asociación de Desarrollo de Videojuegos de la UCA
        \end{alertblock}        
    \end{center}
    
    \begin{columns}[c]
		\column{150pt}
		\begin{block}{La fundamos}
            \begin{itemize}
                \item David Saltares Márquez
                \item Jose Marente Florín
                \item Alberto Cejas Sánchez
                \item Francisco Javier Santacruz López-Cepero
                \item Sebastián Guerrero Selma
            \end{itemize}            
        \end{block}        
		\column{150pt}
		\begin{center}
			\includegraphics[scale=0.2]{img/advuca.png}
		\end{center}
	\end{columns} 
\end{frame}

	
\begin{frame}
	\frametitle{Nuestros objetivos}
	
	\begin{columns}[c]
		\column{75pt}
		\begin{center}
			\includegraphics[scale=0.3]{img/link.png}
		\end{center}
		\column{225pt}
		
		\begin{center}
			Este taller es sólo el principio
		\end{center}		
		
		\begin{block}{Pretendemos}
            \begin{itemize}
                \item Formar grupos de desarrollo de videojuegos
				\item Ofrecer documentación sobre creación de videojuegos
				\item Colaborar con asignaturas de la carrera
				\item Trabajo con grupos interdisciplinares (programación, grafismo, sonido...)
				\item ¡Divertirnos!
            \end{itemize}            
        \end{block}        
	\end{columns}
\end{frame}

\begin{frame}
	\frametitle{Inscríbete}
	
	\begin{columns}[c]
		\column{150pt}
		\begin{block}{Sólo necesitas}
            \begin{itemize}
                \item Matrícula de la UCA
				\item Datos personales
				\item ¡Muchas ganas de aprender!
            \end{itemize}            
        \end{block}        
		\column{150pt}
		\begin{center}
			\includegraphics[scale=0.22]{img/iwantyou.jpg}
		\end{center}
	\end{columns} 
	
\end{frame}

%	======= DISEÑO DE VIDEOJUEGOS =======

\section{Diseño de Videojuegos}

\begin{frame}
	\frametitle{El equipo}
	
	Puede que tus primeros juegos los desarrolles de forma individual
	pero los equipos suelen componerse de:
		
	\begin{columns}[c]
	\column{200pt}	
	
	\begin{block}{Componentes de un equipo}
		\begin{itemize}
			\item Diseñadores
			\item Ingenieros y programadores
			\item Artistas 2D
			\item Artistas 3D
			\item Sonido
			\item Otro equipo: soporte, marketing etc.
		\end{itemize}            
	\end{block}
	
	\column{100pt}
	\begin{center}
	    \includegraphics[scale=0.3]{img/collaborate.jpg}
	\end{center}
	
	
	\end{columns}
\end{frame}

\begin{frame}
	\frametitle{El proceso}
	
	Aunque te de una pereza tremenda deberías mantener una mínima
	disciplina o tu proyecto se quedará en una ¿buena? idea.
		
	\begin{block}{Pasos}
		\begin{itemize}
			\item Definir el juego (tipo, jugabilidad, personajes...)
			\item Ingeniería del Software (¿acaso creías que no valía para nada?)
			\item Implementación
			\item Documentación (¡no lo dejes para el final!)
			\item Muchas pruebas
			\item Lanzamiento
		\end{itemize}            
	\end{block}        
	
\end{frame}

\begin{frame}
	\frametitle{¡Importante!}
	
	\begin{center}
		Recuerda, no creas que tu primeros proyectos serán algo así:
	\end{center}	
	
	\begin{center}
		\includegraphics[scale=0.2]{img/codmw2.jpg}
	\end{center}	
\end{frame}


\begin{frame}
	\frametitle{Probablemente sean así}
	\begin{columns}[c]
	\column{150pt}	
	\begin{center}
	    \includegraphics[scale=0.15]{img/robinson.png}

	    \includegraphics[scale=0.13]{img/spacepenguin.png}
	\end{center}
	
	\column{150pt}
	\begin{center}
	    \includegraphics[scale=0.15]{img/grannysbloodbath.png}
    
	    \includegraphics[scale=0.15]{img/magicduel.png}
	\end{center}
	
	\end{columns}
		
\end{frame}
\begin{frame}
	\frametitle{Game Loop}
	
	\begin{center}
		Todos los juegos siguen la siguiente estructura, \textbf{\emph{Game Loop}}:
	\end{center}	
	
	\begin{center}
		\includegraphics[scale=0.25]{img/gameloop.png}
	\end{center}	
\end{frame}

\subsection{Inicialización}

\subsection{Actualizar}

\subsection{Dibujar}

\subsection{Cierre}

\section{Pong}

\section{SDL}

\section{Implementación}

\section{Ampliando}

\section{¿Y ahora qué?}

\section{Opiniones}

\end{document}
