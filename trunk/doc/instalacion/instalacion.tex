% Tipo de documento
\documentclass[16pt,spanish]{article}

% Ruta a la plantilla
\def \plpath{.}

% Paquetes
% Importaciones y paquetes %%%%%%%%%%%%%%%%%%%%%%%%%%%%%%%%%%%%%%

% Codificación
\usepackage[utf8]{inputenc}
\usepackage[spanish,activeacute]{babel}

% Paquetes extras
\usepackage{listings} % Trozos de código
\usepackage{graphicx} % Imágenes
\usepackage{fancyhdr} % Cabeceras
\usepackage[left=2cm,top=2.5cm,right=2cm,bottom=2.5cm]{geometry}

% Configuración  %%%%%%%%%%%%%%%%%%%%%%%%%%%%%%%%%%%%%%%%%%%%%%%%%

\lstset{language=C++,
	showstringspaces=true}



%  Datos del documento a Editar %%%%%%%%%%%%%%%%%%%%%%%%%%%%%%%%%%%

% Título del documento
\def \titlename{Introducción al Desarrollo de Videojuegos con un Pong} 

\def \subtitulo{Instalación de SDL en Linux y en Windows}

% Autores separados por \and
\def \authorname{David Saltares Márquez}

% Versión de la revisión (En blanco para documentos nuevos)
\def \revname{\ }

% Fecha (En blanco para la fecha de hoy)
\def \datename{}

% Configuración  %%%%%%%%%%%%%%%%%%%%%%%%%%%%%%%%%%%%%%%%%%%%%%%%%

\title{\titlename}
\author{\authorname}
%\date{\datename}

% Cabecera y pie del documento
\pagestyle{fancy}
\renewcommand{\headrulewidth}{0.2pt}
\fancyhead[HC]{ {\footnotesize \titlename} }
\fancyhead[FR]{ {\footnotesize \thepage} }


%%%%%%%%%%%%%%%%%%%%%%%%%%%%%%%%%%%%%%%%%%%%%%%%%%%%%%%%%%%%%%%%%

\begin{document}

% Portada %%%%%%%%%%%%%%%%%%%%%%%%%%%%%%%%%%%%%%%%%%%%%%%%%

% Fichero con la portada %%%%%%

\thispagestyle{empty}
\begin{picture}(0,0)
	\put(115,-35){\includegraphics[scale=0.42]{\plpath/img/cabecera.png}}
\end{picture}\\[4cm]
	
	\begin{center}
		\makeatletter
		{\bf {\Huge \@title}}
		\\[3cm]
		{\bf {\LARGE \subtitulo}}
		\\[2cm]
		\@date\\
		{\footnotesize \revname}
		\\[9cm]
		\begin{tabular}[t]{c} \@author \end{tabular}
		\makeatother
	\end{center}

\cleardoublepage




% Índice %%%%%%%%%%%%%%%%%%%%%%%%%%%%%%%%%%%%%%%%%%%%%%%%%%

%\tableofcontents
%\cleardoublepage

% Cambios en esta Revisión %%%%%%%%%%%%%%%%%%%%%%%%%%%%%%%%%%%%%%
%\section{Cambios}

%\paragraph{}
%Cambios con respecto a versiones anteriores del documento.

%\begin{itemize}
%	\item {\bf Revision 1} 
%		\begin{itemize}
%			\item Cambio 1
%			\item Cambio 2
%		\end{itemize}
%\end{itemize}

%%%%%%%%%%%%%%%%%%%%%%%%%%%%%%%%%%%%%%%%%%%%%%%%%%%%%%%%%%%%%%%%%

\section{Instalación de la biblioteca SDL en GNU/Linux}
\label{sec:linux}

\paragraph{}
Suponemos que tienes una distribución basada en Debian como Ubuntu o Guadalinex
por lo que dispones de \emph{apt-get}. Para instalar SDL y las herramientas
que utilizaremos en el taller, abre una terminal y ejecuta:

\begin{verbatim}
sudo apt-get install gcc make libsdl1.2-dev libsdl-image1.2-dev \
                     libsdl-ttf2.0-dev
\end{verbatim}

\paragraph{}
En sistemas no basados en Debian los paquetes no varían, simplemente habría
que cambiar el nombre del programa para instalar paquetes. Por ejemplo:
\emph{yum}.

\section{Instalación de la biblioteca SDL en Windows}

\paragraph{}
El proceso de instalación de SDL en Windows es bastante más largo que en GNU/Linux
(quizás eso sea una señal para que cambies de sistema operativo). Primero
instalaremos un entorno de desarrollo integrado (IDE) libre llamado CodeBlocks
y más tarde procederemos con SDL. Debes saber que CodeBlocks utiliza el compilador
GCC (a través de MinGW) por lo que trabajaremos aproximadamente sobre lo mismo.

\subsection{CodeBlocks}

\begin{itemize}
	\item Acude a la sección de descargas de la página official de Code Blocks
	y selecciona la versión para Windows que incluye MinGW:
	\begin{verbatim}
	http://www.codeblocks.org/downloads/26#windows
	\end{verbatim}
	\item Ejecuta el instalador, dale los permisos oportunos y elije las opciones
	por defecto.
\end{itemize}

\subsection{SDL}

\paragraph{}
Muy bien, tú lo has querido, has optado por instalar SDL en Windows. Sigue estas
\textbf{largas} instrucciones con cuidado si no quieres tener problemas
a la hora de compilar tu proyecto.

\subsubsection{SDL básico}

\begin{itemize}
	\item Descarga el paquete de ejecución de SDL:
	\begin{verbatim}
	http://www.libsdl.org/release/SDL-1.2.14-win32.zip
	\end{verbatim}
	\item Descomprime el fichero
	\item Copia el fichero \emph{SDL.dll} en la carpeta de instalación
	de Mingw, probablemente en \emph{C:/Mingw/bin}.
	\item Descarga el paquete de desarrollo de SDL desde:
	\begin{verbatim}
	http://www.libsdl.org/release/SDL-devel-1.2.14-mingw32.tar.gz
	\end{verbatim}
	\item Descomprime el fichero.
	\item Copia el contenido del directorio \emph{lib} en el directorio
	de Mingw. Seguramente quede: \emph{C:/Mingw/lib}.
	\item Copia el directorio \emph{SDL} situado dentro de \emph{include}
	a la instalación de Mingw. Debe quedar \emph{C:/Mingw/include/SDL}.
\end{itemize}

\subsubsection{SDL Image}

\subsubsection{SDL TTF}

\end{document}
