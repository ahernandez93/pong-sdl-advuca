\documentclass{beamer}

\usepackage[utf8]{inputenc}
\usepackage[spanish,activeacute]{babel}
\usepackage{listings} % Trozos de código
\usepackage{graphicx} % Imágenes
\usepackage{fancyhdr} % Cabeceras
\usepackage{listings}

\usetheme{Darmstadt}
\usefonttheme{structuresmallcapsserif} 


\title{Desarrollo de un Pong en C con SDL}
\author{David Saltares Márquez \and Alberto Cejas Sánchez}


\AtBeginSection[]
{
  \begin{frame}<beamer>
    \frametitle{Índice}
    \tableofcontents[currentsection,currentsubsection]
  \end{frame}

}

\begin{document}

\begin{frame}
	\titlepage
	
    \begin{picture}(0,0)
        \put(10,-5){\includegraphics[scale=0.1]{img/advuca.png}}
    \end{picture}
    
    \begin{picture}(0,0)
        \put(230,-15){\includegraphics[scale=0.5]{img/cartel.jpg}}
    \end{picture}
    
    \begin{center}
        \includegraphics[scale=1]{img/cc.png}
    \end{center}    
\end{frame}

\begin{frame}
	\frametitle{Índice de contenidos}
	\tableofcontents
\end{frame}


%	======= ADVUCA =======

\section{ADVUCA}

\begin{frame}
    \frametitle{¿Pero qué es la ADVUCA?}
    
    \begin{center}
        \begin{alertblock}{¡Es un acrónimo!}
            \emph{ADVUCA} $\rightarrow$ Asociación de Desarrollo de Videojuegos de la UCA
        \end{alertblock}        
    \end{center}
    
    \begin{columns}[c]
		\column{150pt}
		\begin{block}{La fundamos}
            \begin{itemize}
                \item David Saltares Márquez
                \item Jose Marente Florín
                \item Alberto Cejas Sánchez
                \item Francisco Javier Santacruz López-Cepero
                \item Sebastián Guerrero Selma
            \end{itemize}            
        \end{block}        
		\column{150pt}
		\begin{center}
			\includegraphics[scale=0.2]{img/advuca.png}
		\end{center}
	\end{columns} 
\end{frame}

	
\begin{frame}
	\frametitle{Nuestros objetivos}
	
	\begin{columns}[c]
		\column{75pt}
		\begin{center}
			\includegraphics[scale=0.3]{img/link.png}
		\end{center}
		\column{225pt}
		
		\begin{center}
			Este taller es sólo el principio
		\end{center}		
		
		\begin{block}{Pretendemos}
            \begin{itemize}
                \item Formar grupos de desarrollo de videojuegos
				\item Ofrecer documentación sobre creación de videojuegos
				\item Colaborar con asignaturas de la carrera
				\item Trabajo con grupos interdisciplinares (programación, grafismo, sonido...)
				\item ¡Divertirnos!
            \end{itemize}            
        \end{block}        
	\end{columns}
\end{frame}

\begin{frame}
	\frametitle{Inscríbete}
	
	\begin{columns}[c]
		\column{150pt}
		\begin{block}{Sólo necesitas}
            \begin{itemize}
                \item Matrícula de la UCA
				\item Datos personales
				\item ¡Muchas ganas de aprender!
            \end{itemize}            
        \end{block}        
		\column{150pt}
		\begin{center}
			\includegraphics[scale=0.22]{img/iwantyou.jpg}
		\end{center}
	\end{columns} 
	
\end{frame}


%	======= DISEÑO DE VIDEOJUEGOS =======

\section{Videojuegos}
\begin{frame}
	\frametitle{El equipo}
	
	Puede que tus primeros juegos los desarrolles de forma individual,
	pero los equipos suelen componerse de:
		
	\begin{columns}[c]
	\column{200pt}	
	
	\begin{block}{Componentes de un equipo}
		\begin{itemize}
			\item Diseñadores
			\item Ingenieros y programadores
			\item Artistas 2D
			\item Artistas 3D
			\item Sonido
			\item Otro equipo: soporte, marketing etc.
		\end{itemize}            
	\end{block}
	
	\column{100pt}
	\begin{center}
	    \includegraphics[scale=0.3]{img/collaborate.jpg}
	\end{center}
	
	
	\end{columns}
\end{frame}

\begin{frame}
	\frametitle{El proceso}
	
	Aunque te de una pereza tremenda deberías mantener una mínima
	disciplina o tu proyecto se quedará en una ¿buena? idea.
		
	\begin{block}{Pasos}
		\begin{itemize}
			\item Definir el juego (tipo, jugabilidad, personajes...)
			\item Ingeniería del Software (¿acaso creías que no valía para nada?)
			\begin{itemize}
			    \item Lo más importante es planificarse bien
			\end{itemize}
			\item Implementación
			\item Documentación (¡no lo dejes para el final!)
			\begin{itemize}
			    \item Hay técnicas de documentación automática fáciles y productivas
			\end{itemize}
			\item Muchas pruebas
			\item Lanzamiento
		\end{itemize}            
	\end{block}        
	
\end{frame}

\begin{frame}
	\frametitle{¡Importante!}
	
	\begin{center}
		Recuerda, no creas que tu primeros proyectos serán algo así:
	\end{center}	
	
	\begin{center}
		\includegraphics[scale=0.2]{img/codmw2.jpg}
	\end{center}
	
	\begin{center}
		Serán como los que hacemos en clase...
	\end{center}	
\end{frame}


\begin{frame}
	\frametitle{Probablemente sean así}

	\begin{center}
	Robinson 2.0
	
	    \includegraphics[scale=0.26]{img/robinson.png}
	    
	http://robinson.forja.rediris.es
	\end{center}
\end{frame}

\begin{frame}
	\frametitle{Probablemente sean así}

	\begin{center}
	Space Penguin
	
	    \includegraphics[scale=0.26]{img/spacepenguin.png}
	    
	http://spacepenguinproject.wordpress.com
	\end{center}
\end{frame}

\begin{frame}
	\frametitle{Probablemente sean así}

	\begin{center}
	Granny's Bloodbath
	
	    \includegraphics[scale=0.26]{img/grannysbloodbath.png}
	    
	http://grannysbloodbath.wordpress.com
	\end{center}
\end{frame}

\begin{frame}
	\frametitle{Probablemente sean así}

	\begin{center}
	Magic Duel
	
	    \includegraphics[scale=0.22]{img/magicduel.png}
	    
	http://meydey.es/magicduel
	\end{center}
\end{frame}

\begin{frame}
	\frametitle{Diseño}
	
	\begin{center}
		Antes de lanzarse a programar hay que diseñar el juego
	\end{center}	
	
	\begin{columns}
	
	\column{175pt}
	\begin{block}{Modelado del mundo}
		\begin{itemize}
			\item Planteamiento y concepto
			\item Género
			\item Mecánicas
			\item Estructura de los niveles
			\item Personajes, enemigos y objetos
			\item Público objetivo
		\end{itemize}            
	\end{block}
	
	
	\column{125pt}
	\begin{center}
		\includegraphics[scale=0.2]{img/assassins.jpg}
	\end{center}	
	
	
	\end{columns}
\end{frame}


\begin{frame}
	\frametitle{Game Loop}
	
	\begin{center}
		Todos los juegos siguen esta estructura, \textbf{\emph{Game Loop}}:
	\end{center}	
	
	\begin{columns}
	
	\column{125pt}
	\begin{center}
		\includegraphics[scale=0.25]{img/gameloop.png}
	\end{center}	
	
	\column{175pt}
	\begin{alertblock}{Frames y movimiento}
		\begin{itemize}
			\item \emph{Actualizar}: los elementos reaccionan a su entorno y al usuario.
			\item \emph{Dibujar}: la escena se dibuja en pantalla.
			\item Si cada segundo se producen 25 iteraciones (25fps), tendremos sensación de movimiento.
			\item Guardamos los elementos del juego en variables.
		\end{itemize}            
	\end{alertblock}
	
	\end{columns}
\end{frame}

\begin{frame}
	\frametitle{Game Loop - Inicialización}
	
	Durante la etapa de inicialización se llevan a cabo las siguientes
	tareas:
	
	\begin{columns}[c]
	\column{200pt}
		
	\begin{block}{Pasos}
		\begin{itemize}
			\item Inicializar el motor y las librerías que usemos
			\item Cargar ficheros de configuración
			\item Dejar todo listo para empezar a trabajar
		\end{itemize}            
	\end{block}
	
	\column{100pt}
	
	\begin{center}
		\includegraphics[scale=0.25]{img/engine.png}
	\end{center}	
	
	\end{columns}
	
\end{frame}

\begin{frame}
	\frametitle{Game Loop - Actualizar}
	
	Cada vez que actualizamos tenemos en cuenta:
	
	\begin{columns}[c]
	\column{175pt}
		
	\begin{block}{Actualizar}
		\begin{itemize}
			\item Entrada del usuario
			\item Red
			\item Movimiento y acciones del/los personaje/s
			\item Movimiento y acciones del/los enemigo/s, Inteligencia Artificial
			\item Colisiones, física
			\item Eventos aleatorios o asíncronos
		\end{itemize}            
	\end{block}
	
	\column{125pt}
	
	\begin{center}
		\includegraphics[scale=0.25]{img/grannycolision.png}
	\end{center}	
	
	\end{columns}
	
\end{frame}

\begin{frame}
	\frametitle{Game Loop - Dibujar}
	
	La fase de dibujado en un juego 2D:
	
	\begin{columns}[c]
	\column{175pt}
		
	\begin{block}{Dibujar}
		\begin{itemize}
			\item Dibujamos por capas, como The Gimp o Photoshop
			\item Componemos la escena por piezas
			\item Suele hacerse en orden (lejano-cercano)
			\item Algunas librerías permiten z-order
		\end{itemize}            
	\end{block}
	
	\column{125pt}
	
	\begin{center}
		\includegraphics[scale=0.4]{img/pantalla.png}
		
		\includegraphics[scale=0.4]{img/elementos.png}
		
		\includegraphics[scale=0.4]{img/composicion.png}
	\end{center}	
	
	\end{columns}
	
\end{frame}

\begin{frame}
	\frametitle{Game Loop - Cierre}
	
	¡Hay que recoger antes de salir!
	
	\begin{columns}[c]
	\column{175pt}
		
	\begin{block}{Cierre}
		\begin{itemize}
			\item Liberar todos los elementos del juego
			\item Cerrar las librerías en uso
			\item ¡Prevén las fugas de memoria!
		\end{itemize}            
	\end{block}
	
	\column{125pt}
	
	\begin{center}
		\includegraphics[scale=0.6]{img/tidy.jpg}
	\end{center}	
	
	\end{columns}
	
	\begin{center}
	    ¡Los descuidos de memoria son muy peligrosos!
	\end{center}
	
\end{frame}


\section{Pong}
\begin{frame}
	\frametitle{Pong original}
	
    \begin{center}
        \textbf{La leyenda}
    \end{center}
	
    \begin{center}
		\includegraphics[scale=0.4]{img/pong-original.png}
	\end{center}	

\end{frame}

\begin{frame}
	\frametitle{Pong original}
	
	Para ir abriendo boca:
	
	\begin{columns}[c]
	\column{175pt}
		
	\begin{block}{Pong Facts}
		\begin{itemize}
            \item Uno de los primeros videojuegos de la historia.
			\item Creado por Nolan Bushnell en 1972 para Atari.
            \item Atari demandó a muchísimas copias de Pong.
            \item Tuvo un grandioso éxito... y ahora estamos aquí.
		\end{itemize}            
	\end{block}
	
	\column{125pt}
	
	\begin{center}
		\includegraphics[scale=0.3]{img/pong-recreativa.jpg}
	\end{center}	
	
	\end{columns}
	
\end{frame}

\begin{frame}
	\frametitle{Pong - Diseño}
	
	\begin{center}
	Toca diseñar nuestro Pong, ¡manos a la obra!
	\end{center}
	
	\begin{columns}[c]
	\column{175pt}
		
	Recordad	
	
	\begin{block}{Previously...}
		\begin{itemize}
			\item Planteamiento, concepto y género
			\item Personajes, enemigos y objetos
			\item Mecánicas
		\end{itemize}            
	\end{block}
	
	\column{125pt}
	
	\begin{center}
		\includegraphics[scale=0.3]{img/Lamp-256.png}
	\end{center}	
	
	\end{columns}
	
	\begin{center}
	Con más más formalidad en el Game Design Document
	\end{center}
	
\end{frame}

\begin{frame}
	\frametitle{Pong - Diseño - Planteamiento}
	
	\begin{columns}[c]
	\column{175pt}	
	
	\begin{block}{Planteamiento}
		\begin{itemize}
			\item Juego arcade
			\item Simulación muy básica del tenis
			\item Jugador VS Inteligencia Artificial
		\end{itemize}            
	\end{block}
	
	\column{125pt}
	
	\begin{center}
		\includegraphics[scale=0.2]{img/pong-recreativa-2.jpg}
	\end{center}
	
	\begin{center}
	    Esta recreativa derrocha amor
	\end{center}	
	
	
	\end{columns}
	
\end{frame}

\begin{frame}
	\frametitle{Pong - Diseño - Mecánica}
	
	\begin{columns}[c]
	\column{175pt}	
	
	\begin{block}{Palas}
		\begin{itemize}
			\item Tablero rectangular
			\item El jugador controla una pala
			\item La IA controla otra pala
			\item Deben golpear una pelota para marcar en el campo contrario
			\item La IA será invencible (survival)
			\item Cada vez que se devueleve la pelota el jugador consigue un punto
			\item Superación del récord
		\end{itemize}            
	\end{block}
	
	\column{125pt}
	
	\begin{center}
		\includegraphics[scale=0.45]{img/telepong.jpg}
	\end{center}
	
	\begin{center}
	    Dos jugadores en el mismo pad
	\end{center}	
	
	\end{columns}
	
\end{frame}

\begin{frame}
	\frametitle{Pong - Diseño - Jugador}
	
	\begin{columns}[c]
	\column{175pt}	
	
	\begin{block}{Jugador humano}
		\begin{itemize}
			\item Movimiento: arriba y abajo
			\item Sin salirse del tablero
		\end{itemize}            
	\end{block}
	
	\column{125pt}
	
	\begin{center}
		\includegraphics[scale=0.6]{img/wiipong.png}
	\end{center}	
	
	\begin{center}
	Estas \textbf{NO} serán nuestras palas
	\end{center}
	
	\end{columns}
	
\end{frame}

\begin{frame}
	\frametitle{Pong - Diseño - Bola}
	
	\begin{columns}[c]
	\column{175pt}	
	
	\begin{block}{Bola}
		\begin{itemize}
			\item Movimiento rectilíneo uniforme
			\item Rebotes en palas y bordes
			\item No tendremos en cuenta ángulos de choque
		\end{itemize}            
	\end{block}
	
	\column{125pt}
	
	\begin{center}
		\includegraphics[scale=0.4]{img/pelota.png}
	\end{center}	
	
	\begin{center}
	    Aún no tendremos que desempolvar los libros de física de la ESO
	\end{center}
	
	\end{columns}
	
\end{frame}



\section{SDL}
\begin{frame}
    \frametitle{SD... ¿qué?}
    
    
    \begin{columns}[c]
		\column{220pt}
		\begin{block}{¿Qué es la SDL?}
            \begin{itemize}
                \item Librería multimedia para aplicaciones 2D
                \item Escrita en lenguaje C
                \item Compatible con multitud de lenguajes (C, C++, Python...)
                \item Multiplataforma (Linux, Windows, Mac, PSP...)
                \item Completamente libre (LGPL)
            \end{itemize}            
        \end{block}
        
		\column{100pt}
		\begin{center}
			\includegraphics[scale=0.2]{img/sdl.jpeg}
		\end{center}
	\end{columns}
    
    \begin{center}
        Mucha más información en http://osl.uca.es/wikijuegos
    \end{center}
\end{frame}

\begin{frame}
    \frametitle{Características}
    
    
    \begin{columns}[c]
		\column{160pt}
		\begin{block}{Lo que SÍ ofrece SDL}
            \begin{itemize}
                \item Vídeo
                \item Entrada
                \item Sonido
                \item Red
                \item Fuentes
                \item CD-ROM
                \item Control de tiempo
            \end{itemize}            
        \end{block}
        
		\column{160pt}
        
        \begin{alertblock}{Lo que NO ofrece SDL}
            \begin{itemize}
                \item Física
                \item Colisiones
                \item Inteligencia Artificial
                \item Gestión de recursos
                \item Animaciones
                \item GUI
                \item Aceleración por hardware
            \end{itemize}            
        \end{alertblock}
		
	\end{columns} 
\end{frame}


\begin{frame}
    \frametitle{Blitting}
    
    
    \begin{block}{Capas}
        \begin{itemize}
            \item Una SDL\_Surface es una capa sobre la que dibujar
            \item La SDL\_Surface principal representa la pantalla
            \item Cargamos imágenes en una SDL\_Surface
            \item Volcar una superficie sobre otra se llama \textbf{Blitting}
            \item Volcando superficies componemos la escena
        \end{itemize}            
    \end{block}

\end{frame}


\begin{frame}
    \frametitle{Blitting - Ejemplo}

    \begin{center}
		\includegraphics[scale=0.5]{img/Blitting.png}
	\end{center}
    
\end{frame}

\begin{frame}
    \frametitle{El Flipping y el doble buffer}
    
	\begin{block}{Doble buffer}
        \begin{itemize}
            \item Tenemos dos superficies: pantalla y auxiliar
            \item Hacemos blitting sobre la auxiliar
            \item Terminamos de componer la escena
            \item Intercambiamos pantalla y auxiliar
            \item Evitamos un horroroso parpadeo
        \end{itemize}            
    \end{block}
    
\end{frame}

\begin{frame}
    \frametitle{Flipping - Ejemplo}

    \begin{center}
		\includegraphics[scale=2]{img/flipping.png}
	\end{center}
    
\end{frame}


\section{Implementación}
\begin{frame}
	\frametitle{Pong - Paso 1}
	
    \begin{center}
        \textbf{Resultado}
    \end{center}
	
    \begin{center}
		\includegraphics[scale=0.4]{img/pong-advuca-1.png}
	\end{center}	

\end{frame}

\begin{frame}
	\frametitle{Pong - Paso 2}
	
    \begin{center}
        \textbf{Resultado}
    \end{center}
	
    \begin{center}
		\includegraphics[scale=0.4]{img/pong-advuca-2.png}
	\end{center}	

\end{frame}

\begin{frame}
	\frametitle{Pong - Paso 3}
	
    \begin{center}
        \textbf{Resultado}
    \end{center}
	
    \begin{center}
		\includegraphics[scale=0.4]{img/pong-advuca-3.png}
	\end{center}	

\end{frame}

\begin{frame}
	\frametitle{Pong - Paso 4}
	
    \begin{center}
        \textbf{Resultado}
    \end{center}
	
    \begin{center}
		\includegraphics[scale=0.4]{img/pong-advuca-4.png}
	\end{center}	

\end{frame}

\begin{frame}
	\frametitle{Pong - Paso 5}
	
    \begin{center}
        \textbf{Resultado}
    \end{center}
	
    \begin{center}
		\includegraphics[scale=0.4]{img/pong-advuca-5.png}
	\end{center}	

\end{frame}

\begin{frame}
	\frametitle{Pong - Paso 6}
	
    \begin{center}
        \textbf{Resultado}
    \end{center}
	
    \begin{center}
		\includegraphics[scale=0.4]{img/pong-advuca-6.png}
	\end{center}	

\end{frame}


\section{Ampliación}
\begin{frame}
	\frametitle{Añadidos del Pong}
	
    
	\begin{columns}[c]
		\column{200pt}
		
	\begin{block}{Lo que se puede mejorar}
            \begin{itemize}
                \item Mejores gráficos (¡trabaja siempre con artistas!)
                \item Inteligencia Artificial torpe
                \item Efectos de sonido (rebote, victoria, derrota)
                \item Guardar la puntuación en un fichero
		\item Mejora del rendimiento
            \end{itemize}            
        \end{block}        
		
        \column{100pt}
		\begin{center}
			\includegraphics[scale=0.32]{img/expand.jpg}
		\end{center}
        
	\end{columns} 

\end{frame}


\begin{frame}
	\frametitle{Cambiamos a PyGame}
	
	      
	Nuestro \emph{querido} amigo Jose nos enseñará su Pong (powered by PyGame)
	
	\begin{center}
	    \includegraphics[scale=0.32]{img/pygame.png}
	\end{center}

\end{frame}



\section{Más allá}

%	======= 1) NOW WHAT? =======

\begin{frame}
	\frametitle{Muy bien pero...¿Cómo sigo?}
	
	\begin{center}
		\includegraphics[scale=0.40]{img/nowwhat.jpg}
	\end{center}

\end{frame}

%	======= 2) REQUISITOS =======

\begin{frame}
	\frametitle{¿Qué necesito?}
	
	Como has podido ver desarrollar un videojuego no es tarea fácil, requiere de un proceso de constante aprendizaje, investigación, dedicación y mucha ayuda.
		
	\begin{block}{Recomendaciones}
		\begin{itemize}
			\item \textbf{Motivación}: Hay que dedicarle mucho tiempo.
			\item \textbf{Curiosidad}: Interés por aprender cosas nuevas y mejorar lo que ya sabes.
			\item \textbf{Saber Inglés}: Gran cantidad de los recursos de calidad están en inglés.
			\item \textbf{Contactos}: Trabajo artístico, publicitario, sonido, beta testers...
			\item \textbf{Comunidad}: Lugar de referencia para consultar, compartir, etc.
		\end{itemize}
	\end{block}

\end{frame}

%	======= 3) LENGUAJES =======

\begin{frame}
	\frametitle{Lenguajes Alto Nivel}
	
	Según los conocimientos que tengas, así como los propósitos y metas del juego a desarrollar es importante elegir bien el lenguaje de programación.
		
	\begin{block}{Recomendaciones}
		\begin{itemize}
			\item C/C++/C\#
			\item Python
			\item Java
			\item Lua
			\item Lisp
			\item Ruby
		\end{itemize}
	\end{block}

\end{frame}

%	======= 4) LIBRERÍAS 2D =======

\begin{frame}
	\frametitle{Librerías 2D}
	La mejor forma de adentrarse en el desarrollo de videojuegos.
	\newline
	\begin{columns}[c]
		\column{100pt}
		\begin{block}{Algunos ejemplos destacados}
            \begin{itemize}
							\item Gosu
							\item SDL
							\item Allegro
							\item Pygame
							\item ClanLib
            \end{itemize}            
        \end{block}        
		\column{100pt}
		\begin{center}
			\includegraphics[scale=0.32]{img/allegro.png}
			\newline
			\newline
			\includegraphics[scale=0.22]{img/clanlib.png}
			\newline
			\newline
			\includegraphics[scale=0.22]{img/pygame.png}
		\end{center}
		\column{100pt}
		\begin{center}
			\includegraphics[scale=0.045]{img/sdl.png}
			\newline
			\includegraphics[scale=0.23]{img/gosu.png}
		\end{center}
	\end{columns} 

\end{frame}

%	======= 5) LIBRERÍAS 3D =======

\begin{frame}
	\frametitle{Librerías 3D}
	El salto de calidad... 			\hspace{5.5cm}\includegraphics[scale=0.12]{img/2dto3d.jpg}
	\newline
	\begin{columns}[c]
		\column{150pt}
		\begin{block}{Algunos ejemplos destacados}
            \begin{itemize}
							\item Ogre
							\item IrrLicht
							\item Crystal Space
							\item Panda
            \end{itemize}            
        \end{block}        
		\column{150pt}
		\begin{center}
			\includegraphics[scale=0.30]{img/ogre.png}
			\hspace{0.20cm}
			\includegraphics[scale=0.40]{img/crystal.png}
			\newline
			\newline
			\includegraphics[scale=0.50]{img/panda.png}
			\hspace{0.60cm}
			\includegraphics[scale=1.20]{img/irrlicht.png}
		\end{center}
	\end{columns} 

\end{frame}

%	======= 6) LIBRERÍAS AUXILIARES =======

\begin{frame}
	\frametitle{Otras Librerías y Herramientas}
		
	En muchas ocasiones es necesario apoyarse en otras librerías externas o herramientas que desarrollarán funciones auxiliares.
	\newline
	\begin{columns}[c]
		\column{150pt}
			\includegraphics[scale=0.40]{img/blender.jpg}
		\column{150pt}
		\begin{block}{Algunos ejemplos destacados}
			\begin{itemize}
				\item Física: \emph{ODE, BULLET.}
				\item Gestión de Entrada: \emph{OIS.}
				\item Diseño Artístico: \emph{Gimp, Blender.}
				\newline
			\end{itemize}
		\end{block}
	\end{columns}

\end{frame}

%	======= 7) RECURSOS BIBLIOGRÁFICOS =======

\begin{frame}
	\frametitle{La biblioteca}
	
	\begin{center}
		\includegraphics[scale=0.50]{img/biblio.jpg}
	\end{center}

\end{frame}

%	======= 8) TRABAJO EN EQUIPO =======

\begin{frame}
	\frametitle{Trabajo en Equipo}
	
	\begin{center}
		\includegraphics[scale=0.70]{img/equipo.jpg}
	\end{center}

\end{frame}

%	======= 9) VAYA TELA CON LA UCA: LABOON =======

\begin{frame}
	\frametitle{Algunos ejemplos del potencial existente}
	
	\begin{center}
		\includegraphics[scale=0.25]{img/laboon.png}
	\end{center}

\end{frame}

%	======= 10) VAYA TELA CON LA UCA: FREEPADEL =======

\begin{frame}
	\frametitle{Algunos ejemplos del potencial existente}
	
	\begin{center}
		\includegraphics[scale=0.45]{img/freepadel.png}
	\end{center}

\end{frame}

%	======= 11) VAYA TELA CON LA UCA: AVIONES =======

\begin{frame}
	\frametitle{Algunos ejemplos del potencial existente}
	
	\begin{center}
		\includegraphics[scale=0.30]{img/avion.png}
	\end{center}

\end{frame}

%	======= 12) ADVUCA, TU AMIGO FIEL =======

\begin{frame}
	\frametitle{ADVUCA, tu amigo fiel}
		
	Desde la ADVUCA nos comprometemos a ofrecer ayuda para fomentar desarrollo de proyectos e incentivar el aprendizaje a través de los videojuegos.

	\begin{block}{Algunos ejemplos destacados}
		\begin{itemize}
			\item Formación de grupos de trabajo \emph{(juegos, manuales, etc).}
			\item Difusión.
			\item Consultas.
		\end{itemize}
	\end{block}

	\begin{center}
		\includegraphics[scale=0.50]{img/micro.png}
	\end{center}

\end{frame}

%	======= 13) MUNDO LABORAL =======

\begin{frame}
	\frametitle{El Exteriooooor...}
		
	Después de la UCA, empieza lo bueno...
	\newline
	\begin{columns}[c]
		\column{150pt}
			\begin{block}{Algunas posibilidades}
				\begin{itemize}
					\item Masters.
					\item Desarrolladoras de Videojuegos.
					\item Centros Multimedia.
				\end{itemize}
			\end{block}
		\column{150pt}
			\begin{center}
				\includegraphics[scale=0.15]{img/exterior.jpg}
			\end{center}
	\end{columns}
\end{frame}

%	======= 14) DEJATE DE ROYOS Y NO TE QUEDES MÁS CONMIGO =======

\begin{frame}
	\frametitle{Pero... ¿de verdad?}
	\begin{center}
		\includegraphics[scale=0.20]{img/comandos.jpeg}
	\end{center}
\end{frame}






\section{Opiniones}
\begin{frame}
	\frametitle{¡Gracias!}
	
	\begin{center}
    \huge{¡Muchas gracias por venir!}
    \end{center}
		
    \begin{center}
        \emph{No lancen los tomates todavía}
    \end{center}
    
	\begin{columns}[c]
	\column{200pt}	
	
	http://blog.advuca.com
	
	\begin{block}{Nos interesa vuestra opinión}
        \begin{itemize}
            \item ¿Opinión del taller?
            \item Cosas para mejorar
            \item Temas que interesa tratar
        \end{itemize}
	\end{block}
	
	\begin{center}
	    \textbf{¡Únete a nosotros!}
	\end{center}
	
	\column{100pt}
	\begin{center}
	    \includegraphics[scale=0.3]{img/thumbs-up.jpg}
	\end{center} 
    
    \end{columns}  
	
\end{frame}


\end{document}
